\documentclass[a4paper,11pt,twoside]{ThesisStyle}

\include{formatAndDefs}

\begin{document}

\begin{titlepage}
\begin{center}
\vspace*{2.0cm}
\noindent {\Huge \textbf{This is}} \\
\vspace*{0.2cm}
\noindent {\Huge \textbf{my fancy}} \\
\vspace*{0.2cm}
\noindent {\Huge \textbf{Title}} \\

\vspace*{1.5cm}
\noindent \large {Vorgelegt von\\}
\vspace*{0.1cm}
\noindent \large {\bf Vorname \textsc{Nachname}} \\
\vspace*{0.1cm}
\noindent \large {aus Nimmerland.\\}
\vspace*{1.5cm}

\noindent \large {Von der Fakult\"at IV - Elektrotechnik und Informatik} \\
\vspace*{0.1cm}
\noindent \large {der Technischen Universit\"at Berlin} \\
\vspace*{0.1cm}
\noindent \large {zur Erlangung des akademischen Grades} \\
\vspace*{0.2cm}
\noindent \large {\bf Bachelor/Master of Science} \\
\vspace*{0.1cm}
\noindent \large {\bf - B./M.Sc. -} \\
\vspace*{0.2cm}
\noindent \large {genehmigte Abschlussarbeit.} \\
\vspace*{1.5cm}

\vspace*{0.1cm}
~\\
\begin{tabular}{ll}
      \vspace*{0.1cm}
      \noindent \large{Gutachter :}	& \noindent \large{Prof. Dr.-Ing. Gut \textsc{Achter 1}}\\
      \vspace*{0.1cm}
					& \noindent \large{Prof. Dr.-Ing. Gut \textsc{Achter 2}}\\
	\vspace*{0.1cm}
	\noindent \large{Betreuer :}	& \noindent \large{Prof. Dr.-Ing. Bet \textsc{Reuer}}
\end{tabular}

\vspace*{1.5cm}

\end{center}
\end{titlepage}
\sloppy

\titlepage

\begin{titlepage}
%\begin{center}
\vspace*{5.0cm}
\large{\bf Eidesstattliche Versicherung}\\
\\
Hiermit erkl\"are ich, dass ich die vorliegende Arbeit selbstst\"andig und eigenh\"andig sowie ohne unerlaubte fremde Hilfe und ausschlie\ss lich unter Verwendung der aufgef\"uhrten Quellen und Hilfsmittel angefertigt habe.
\vspace*{1cm}
~\\
\begin{tabular}{lc}
      \vspace*{0.1cm}
      \noindent Berlin, den 05. Juni 2023 & \noindent ............................................ \\
      \vspace*{0.1cm}
					      & \noindent \large{Mohamed Samy \textsc{Elsisi}}
\end{tabular}


%\end{center}
\end{titlepage}
\sloppy

\titlepage


\dominitoc

\pagenumbering{roman}

\cleardoublepage
\begin{vcenterpage}
\noindent\rule[2pt]{\textwidth}{0.5pt}
\begin{center}
{\large\textbf{Abstract\\}}
\end{center}
With the increasing deployment of renewable energy assets, automated condition monitoring solutions are crucial for scaling up wind turbine portfolios. 
This thesis aims to bridge the gap between SCADA signals and SCADA logs in wind turbine condition monitoring by incorporating SCADA log data into normal behavior models. 
By mining SCADA log data, subtle patterns and dependencies can be identified which can enhance the accuracy and robustness of the models. 
Advanced machine learning algorithms, including deep learning and anomaly detection techniques, are employed to detect abnormal behaviors and potential faults. 
The research contributes to improved fault detection, condition assessment, and predictive maintenance strategies, leading to enhanced operational efficiency and reliability of wind turbines.
\noindent\rule[2pt]{\textwidth}{0.5pt}
\end{vcenterpage}

\clearpage
\begin{vcenterpage}
\noindent\rule[2pt]{\textwidth}{0.5pt}
\begin{center}
{\large\textbf{Zusammenfassung\\}}
\end{center}
Mit dem zunehmenden Einsatz von Windkraftanlagen im Bereich der erneuerbaren Energien sind automatisierte Zustandsüberwachungslösungen von entscheidender Bedeutung für die Vergrößerung des Portfolios von Windturbinen. 
Diese Arbeit zielt darauf ab, die Lücke zwischen SCADA-Signalen und SCADA-Logs bei der Zustandsüberwachung von Windkraftanlagen zu schließen, indem SCADA-Logdaten in normale Verhaltensmodelle integriert werden. 
Durch die Auswertung von SCADA-Logdaten können subtile Muster und Abhängigkeiten identifiziert werden, was die Genauigkeit und Robustheit der Modelle verbessert. 
Fortgeschrittene Algorithmen des maschinellen Lernens, einschließlich Deep Learning und Techniken zur Erkennung von Anomalien, werden zur Erkennung von abnormalem Verhalten und potenziellen Fehlern eingesetzt. 
Die Forschung trägt zu einer verbesserten Fehlererkennung, Zustandsbewertung und vorausschauenden Wartungsstrategien bei, was zu einer höheren Betriebseffizienz und Zuverlässigkeit von Windkraftanlagen führt.
\noindent\rule[2pt]{\textwidth}{0.5pt}
\end{vcenterpage}

% \section*{Acknowledgments}

% Last thing to do :-)

\tableofcontents

\mainmatter

\chapter{Introduction}
\label{chap:intro}
\minitoc

\section{Background}
In 2020, renewable energy represented 22.1\% of energy consumed in the EU \cite{Renewable_energy_statistics}. This
percentage is expected to increase drastically in the upcoming years with the target, set by
the European Commission, of at least 32\% by the year 2030 \cite{Renewable_energy_targets}. With the increasing number
of renewable energy assets being deployed every year, automated condition monitoring
solutions are needed for operators to be able to scale up. This need gets more relevant in the
case of operating offshore wind farms, where the cost of maintenance relative to the levelized
cost of energy (LCOE) is significantly higher compared with onshore \cite{SCADA_NBM_Review}.
Several approaches for condition monitoring were developed in the recent years that use
SCADA1 data given its low cost (normally requiring no additional sensors).
One of the methods used for condition monitoring using SCADA data is Normal Behavior
Modelling (NBM). NBM uses the idea of detecting anomalies from normal operation by
empirically modelling a measured parameter, used to reflect the condition of a specific part of
the turbine, based on a training phase (usually during a healthy state of the turbine). During
operation, the difference between the measured and the modelled/predicted signal is used as
indicator for a possible fault. A difference of 0, with some tolerance, reflects normal conditions,
whereas a difference greater or less than 0 reflects changed conditions or failures \cite{SCADA_NBM_Review}.

\section{Motivation}
While NBMs using SCADA data were proven capable of predicting failures \cite{SCADA_NBM_Review}, they are
treated as black box by the operators since they don’t provide any insights regarding the root
cause of the failure. Incorporating SCADA log data2 to NBM could help tackle this problem by
providing some insights to an anomaly detected by the model in case a relevant warning or failure message was logged by the SCADA system around the same time.
It was also shown that incorporating SCADA logs containing information about operation
conditions or control events could help improve the accuracy of the model in case of events
unexplainable by the input signals \cite{Letzgus_Log}.//

Logs were "Never" treated as input feature in NBMs



\chapter{Methods}
\label{chap:methods}
Here, we extensively explain the methods we used and propose to utilize SCADA log messages in wind turbine normal behavior machine learning models.
We start by introducing the dataset used to run the experiments.
Then, we define the concept of normal behavior modeling and the machine learning models and their architecture used in this work.
Finally, we introduce two different approaches to utilizing SCADA logs with ways of using them as input features in machine learning models, as s filter for the SCADA signals,
or to visualize relevant warnings.

\section{Dataset}
In this section, we will describe the dataset used in this work to train, test and validate the models. 
\par We used open-source data published on the \emph{EDP OpenData} web platform \cite{EDP} and made available for research purposes. 
The data was collected from the SCADA systems of five different Vestas wind turbines (Turbine 01, 06, 07, 09 and 11) in the same wind park between the years 2016 and 2017 
and is made up of the following four subsets: \emph{Signals, Logs, Failures and Metmast}. We will, however, only describe three sets since \emph{Metmast} was not used in this work.

\subsection{Signals}
 The \emph{Signals} dataset contains 10-min aggregated data (mean, standard deviation, minimum, and maximum values) collected from the wind turbines' power meters and
 sensors installed at the major components such as gearbox, generator and transformer (see Figure \ref{fig:WTG_Diagram} for a 
 demonstration of a turbine's hardware and the location of major components).These built-in sensors measure quantities such as temperatures, angles, wind and rotational speeds, 
 power production,\dots

 \begin{figure}[H]
  \begin{center}
    \includegraphics[scale=0.3]{Methods/Horizontal-axis-wind-turbine.png}
  \end{center}
  \caption{Diagram of a wind turbine side view with labeled main components. Figure adapted from \cite{WTG_Diagram}}
  \label{fig:WTG_Diagram}
\end{figure}

 This dataset was the most crucial for this work since it provides information that reflects the status of the turbine operation which is needed to perform 
 SCADA-based automated condition monitoring and predictive maintenance.\\
 Table \ref{tab:signals} shows some of the 81 signals included in this dataset.
 \begin{table}[H]
        \centering
    \begin{tabular}{ | m{12em} | m{8cm} | }
    \hline
         \multicolumn{1}{|c|}{\textbf{Type of signal}} & \multicolumn{1}{c|}{\textbf{Signals}} \\
         \hline
         Temperature (\degree C) & Generator, Generator bearings, Hydraulic group oil, Gearbox oil, Gearbox bearing on the high-speed shaft, 
         Nacelle, High Voltage (HV) transformer, Ambient temperature,\dots  \\
         \hline
         Production value & Active power in Wh, Reactive power in VArh, Power according to the grid in kW,\dots \\
         \hline
         Angle (\degree) & Blades pitch angle ($\theta$) \\
    \hline
    \end{tabular}
    \caption{Example signals found in the Signals dataset}
        \label{tab:signals}
\end{table}

 \subsection{Logs}
  In this dataset, events logged by the SCADA system are collected in non-fixed intervals. The events recorded by the system are divided into three categories: Alarm log, 
  Warning log and Operation and System log. According to the VestasOnline Enterprise user manual \cite{voe},  alarms are system notifications that alert operators to 
  an error scenario that has forced a wind turbine to cease normal operation and transition to one of three operational states: Pause, Stop, or 
  Emergency (one of the following three acknowledgments is needed to resume operation: Local acknowledgment 
  from the controller unit of the turbine, Remote acknowledgment from VestasOnline®, or Automatic acknowledgment), 
  whereas warnings are system messages that indicate an irregularity that requires attention but does not cause the turbine 
  to immediately cease normal operation and exit the Run state.
  \begin{table}[H]
          \centering
      \begin{tabular}{|c|c|}
      \hline
          \textbf{Type of log event} & \textbf{Sample log event}  \\
          \hline
          \multirow{2}{12em}{\centering Alarm log} & \emph{"High temperature brake disc"} \\
          & \emph{"High pres offlin:\_\_\_\_RPM/ \_\_\_\degree C"} \\
          \hline
          \multirow{2}{12em}{\centering Warning log} & \emph{"Yaw Position is changed: \_\_\degree"} \\
          & \emph{"Low Battery Nacelle"} \\
          \hline
          \multirow{2}{12em}{\centering Operation and System log} & \emph{"External power ref.:\_\_\_\_kW"} \\
          & \emph{"GearoilCooler \_, gear: \_\_\_\degree C"} \\
      \hline
      \end{tabular}
      \caption{Sample log events found in the Logs dataset}
          \label{tab:metrics}
  \end{table}
\subsection{Failures}
  The Failures dataset contains the history of failures, inspections, or maintenance that occurred in the turbines and was manually recorded by technicians. 
  Each record reports the time of the event, component (e.g., Generator, Hydraulic group,..), and a text description of the failure 
  or event (e.g., "Generator replaced", "Oil leakage in Hub",..).\\ 
  This dataset was used in backtesting to validate the models' capability of detecting failures early.

\clearpage

\section{Normal behavior modeling}
According to Tautz-Weinert and Watson \cite{SCADA_NBM_Review}, Normal Behavior Modeling (NBM) detects anomalies in normal operation by empirically modeling an 
observed parameter based on a training phase. Figure \ref{fig:NBM} depicts the concept of model-based monitoring.
During operation, an anomaly is detected by deducting the value of the modeled signal ($\hat{y}$(t)) from the measured one (y(t))  and 
comparing the residual (e(t)) with a predefined threshold. If the threshold is exceeded, this signal is labeled as an anomaly.
There are two primary approaches for NBM: Full Signal ReConstruction (FSRC), in which only signals other than the target are utilized to predict the target, 
and AutoRegressive with eXogenous Input Modeling (ARX), in which previous values of the target are also employed.\\
Having defined NBM on an abstract level, we demonstrate next the machine learning models we used to generate modeled signals (y(t)) from input signals (x(t)) and 
then explain the anomaly detection approach we used. Given that the structure of the available signals (e.g., the number of signals and their frequency) varies based 
on the turbine's model, manufacturer and sensors installed, we defined the dataset used in this work in the previous section.

\begin{figure}[H]
  \begin{center}
    \includegraphics[scale=0.35]{Methods/NBM.png}
  \end{center}
  \caption{NBM with the input signals from the SCADA system (x(t)), measured signal (y(t)), modeled signal ($\hat{y}$(t)) and resulting error (e(t))}
  \label{fig:NBM}
\end{figure}

\subsection{Machine Learning in NBM}
  From a wide range of machine learning model types, NBM focuses on regression models \cite{Regression}. Regression models are part of the supervised learning family, 
  where the algorithm is trained on labeled data and the input features are mapped to corresponding output labels. 
  As opposed to classification models, where the algorithm predicts \emph{classes}, a regression model predicts numerical \emph{values} (dependent variables) from the input 
  features (independent variables).\\
  According to Tautz-Weinert and Watson \cite{SCADA_NBM_Review}, there are mainly three types of NBM regression models used in the research field: \emph{Linear and polynomial models},
  \emph{Artificial Neural Networks (ANNs)} and \emph{Fuzzy Systems}. Given their simplicity, we used linear models in the early phases of this work. Later on, we started using 
  ANNs for their capability of capturing non-linear dependencies in the data. We define these two types of models in detail in the next subsections.
  A fuzzy system \cite{Neuro_fuzzy} is an artificial intelligence system that employs fuzzy logic \cite{fuzzy_sets}. Fuzzy logic is a mathematical framework for 
  dealing with uncertainty and imprecision.
  The input and output variables in a fuzzy system are represented by fuzzy sets, which are collections of values with degrees of membership rather than tight boundaries. 
  The associations between the input variables and the output variables are then specified using fuzzy rules. 
  These rules are often represented as "if-then" statements, with the "if" section defining the input conditions and the "then" part defining the output actions.
  Training fuzzy systems was not within the scope of this work. However, we propose testing our methods on them in the future works section (see \ref{sec:future_works}).

  \subsubsection{Linear regression}
    Sir Francis Galton proposed the idea of linear regression in 1894 \cite{Natural_Inheritance}.
    Linear regression is used for analyzing the linear relationship between one or more independent variables and a dependent variable.
    The dependent variable must be continuous, whereas the independent variables can be continuous or categorical. For a dependent variable $Y$ and a set of $n$ independent
    variables $X_1$ through $X_n$, the linear regression equation is defined as follows:
    \begin{equation}
      Y = m_1X_1 + m_2X_2 + ... + m_nX_n + C
    \end{equation}
    where $m_1$ through $m_n$ and $C$ are constants. Figure \ref{fig:linear_regression} shows an example of linear regression for a single independent variable.

    \begin{figure}[H]
      \begin{center}
        \includegraphics[scale=0.7]{Methods/Linear_regression.png}
      \end{center}
      \caption{Example linear regression with one dependent variable: $Y = mX + b$, where m and b are constants}
      \label{fig:linear_regression}
  \end{figure}

    When the relationship between the dependent variable and the independent variables is assumed to be linear, 
    linear regression is usually used. Linear regression is easy to use and understand, and it can be used to make predictions or find relationships between variables.\\
    In the example of normal behavior modeling for a wind turbine component, the dependent variable can be defined as the component's temperature 
    and the independent variables as a set of weather and turbine conditions measures (e.g., wind speed, ambient temperature, production value, other components' temperatures,..) 
    that have either a direct or indirect effect on the target component.
    NBM in its most basic form is based on linear or polynomial models \cite{SCADA_NBM_Review}. 
    Garlick et al. \cite{Garlick} employed a linear ARX model to detect generator bearings failures in bearing temperature measurements.
    Schlechtingen and Santos \cite{Schlechtingen} developed an FSRC linear condition monitoring model for the generator bearings' temperature.\\
    Although multiple linear regression models were also shown capable of fitting the data with high accuracy in many other applications (e.g., \cite{Linear_Regression_Example_1}), 
    they are, by definition, not capable of capturing more complex non-linear dependencies. In addition to that, linear regression may not be appropriate when there are a significant 
    number of independent variables. Artificial Neural Networks (ANNs) may be a better approach in these situations.


  \subsubsection{Artificial Neural Networks}
    Artificial Neural Networks (ANNs) are computational models that are inspired by the structure and function of biological neural networks in the brain \cite{NN_foundation}.
    They are made up of interconnected nodes (artificial neurons) that process and send data. 
    Pattern recognition, computer vision, natural language processing, and robotics have all made extensive use of ANNs 
    (for a comprehensive review of deep learning and neural networks, see \cite{Deep_learning_overview}, \cite{Goodfellow}).
    An artificial neuron, also known as a perceptron, is the fundamental building unit of a neural network. It is a mathematical function that accepts one or more input values 
    and outputs a single value \cite{Perceptron}.
    The input values are weighted, and the neuron applies an activation function to the total of the weighted inputs. The output value is subsequently passed on to the network's 
    other neurons. The activation function determines the neuron's output based on the input value(s) and weights.
    For a set of inputs $X_1$ through $X_n$, weights $w_1$ through $w_n$ and activation function $f$, the output of a perceptron $Y$ is calculcated as follows:
    \begin{equation}
      Y = f(\sum_{i=1}^n w_iX_i)
    \end{equation}
    The sigmoid function, the rectified linear unit (ReLU) function, and the hyperbolic tangent function are examples of common activation functions.
    Figure \ref{fig:Perceptron} shows a diagram of a perceptron.

    \begin{figure}[H]
      \begin{center}
        \includegraphics[scale=0.4]{Methods/Perceptron.png}
      \end{center}
      \caption{Example perceptron with three inputs}
      \label{fig:Perceptron}
    \end{figure}

    In the context of deep learning, an ANN consists of one input layer, one output layer and one or more \emph{hidden} layers.\\
    A hidden layer is a layer of neurons that is not connected directly to either the input or output layers. 
    It is referred to as "hidden" because its neurons are not visible to the outside world, implying that its calculations are not directly apparent from input or output.\\
    Information goes from the input layer, through one or more hidden layers, and then to the output layer in a \emph{feedforward} neural network, which is a type of ANN. 
    Each layer of neurons computes on the input data and sends the results to the next layer. The hidden layers extract and alter information from input data that can be 
    utilized to make predictions or choices.\\
    The number of hidden layers in an ANN is a hyperparameter that can be tuned during the training process.
    The number of hidden layers and neurons in each layer is determined by the task's complexity, the amount of accessible data, and the required level of accuracy.\\
    After obtaining better results with it compared to linear regression (see Experiment \ref{exp:I}), we decided to train the normal behavior models
    on a feed-forward neural network having the architecture shown in Fig. \ref{fig:MLP} and ReLU (firstly introduced by Fukushima \cite{Fukushima}) as an activation function.

    \begin{figure}[H]
      \begin{center}
        \includegraphics[scale=0.4]{Methods/MLP_cropped.png}
      \end{center}
      \caption{Architecture of normal behavior neural network model used in this work. \\
      \emph{The input layer shape will vary based on the experiment and the number of input features.}}
      \label{fig:MLP}
    \end{figure}

  \subsection{Anomaly detection}
    The main idea behind training and improving normal behavior models is to allow our models to detect anomalies more accurately.
    An anomaly is defined as an occurrence or observation that differs from what is expected, usual, or typical. 
    An anomaly is commonly referred to as an outlier or an uncommon trend in data in numerous domains such as statistics, data analysis, and security.
    By comparing the observed data to a reference set, such as historical data or a pre-defined model, anomalies can be found.
    Positive and negative anomalies are also possible. Depending on the context, positive anomalies could suggest that something is performing better than predicted and 
    negative anomalies indicate that something is underperforming (e.g., in the context of a company's sales figures), or the other way around. In the context of wind
    turbine condition monitoring and when mainly monitoring temperatures of the system, we focus on positive anomalies because a component that is 
    overheating---due to wear and tear, oil leakage, faulty fan,\dots---is likely to fail. There is, however, no unified method in the research field to identify a data point 
    as an anomaly. Brandao et al. (\cite{Brandao_1}, \cite{Brandao_2}) used a fixed value of the mean absolute error as an anomaly threshold in their 
    gearbox and generator fault detection model, even though this number was particular and no longer valid following maintenance procedures. 
    Schlechtingen and Santos \cite{Schlechtingen} used daily average prediction errors in generator bearings temperature to trigger alarms. 
    Zhang and Wang \cite{Zhang_Wang} used a hard threshold of 1.5\degree C for the residual to identify anomalies in the main shaft rear bearing temperature.
    Bangalore and Tjernberg (\cite{Bangalore_1}, \cite{Bangalore_2}, \cite{Bangalore_3}) used a Mahalanobis distance to compare residual and target distributions from 
    the training period to find anomalies in gearbox bearings temperatures. The Mahalanobis distance was averaged over three days and compared to a training result-defined threshold.\\

    \par As there is no standard way to identify anomalies in temperatures in the context of condition monitoring for wind turbines using normal behavior models, we experimented 
    with several methods to do that and, finally, decided to set the anomaly threshold to the maximum prediction error seen in the training period. This way it is 
    guaranteed that the normal behavior models will not label any data point in the training dataset as an anomaly (complying with the assumption that the turbine was operating 
    in a healthy state during the training phase of the model) while having the threshold dynamically set based on the setup (e.g., 
    input and output features, training period, condition of the turbine during the training phase,\dots) without having to incorporate any domain knowledge related to the 
    specific component to-be-monitored. This also helped better compare different architectures of normal behavior models and the effect of incorporating the proposed log features, 
    not only in terms of prediction accuracy but also in terms of the quality and frequency of anomalies identified (a model that better fits the training data will have a 
    tighter anomaly threshold).

    \subsubsection{Anomaly vs Alarm}
      In our approach, we differentiate between \emph{Anomalies} and \emph{Alarms}. An anomaly is a data point that deviates from "normal", whereas an alarm is a proactive 
      way of communication that gets triggered when the operator's attention is urgently needed. The reason why we propose not to send an alarm every time an anomaly is 
      detected by the system is that we want our system to limit the number of false alarms as they are costly and counterproductive.\\
      As opposed to anomalies, which are tracked on a 10-min basis, we base alarms on daily events. If the number of anomalies found from the start of a day up until a given 
      point in time exceeds a certain threshold, an alarm is triggered. We set the \emph{alarm threshold} to the 99\textsuperscript{th} percentile of the distribution of the number of 
      anomalies that occurred per day during the training period when using an \emph{anomaly threshold} set to the 99\textsuperscript{th} percentile of the distribution of the 
      training prediction errors. To summarize, an alarm can be defined as an anomaly in the number of system anomalies found per day.\\
      (TODO: Maybe soma visualization is needed here?)
  
  \subsection{Feature selection}
    The way the independent variables are chosen is usually done by measuring the correlation coefficients between available features in a
    dataset and the target feature and then selecting the features having a high correlation coefficient. Depending on the problem setting, other features can be also considered 
    based on domain knowledge, especially when dealing with a mechanical system as in the case of this work. A good example of this would be the incorporation of 
    the ambient temperature measurement as an input feature---even if it does not highly correlate with the target feature---to make sure that your model generalizes when 
    trying to predict a component's temperature throughout the year, by considering the effect of seasonality 
    (temperatures are expected to be higher in summer than in winter).\\
    In this work, we selected input features based on both domain knowledge and correlation coefficients. We used Kendall's method to measure the rank correlation \cite{Kendall}.
    In contrast to Pearson's correlation coefficient, Kendall's rank correlation can capture both linear and non-linear dependency between two variables by 
    measuring the monotonic relationship. In addition to that, variables don't have to be normally distributed when using Kendall's method.\\
    (TODO: list features selected)

\clearpage
  
\section{Log analysis}
In this section, we will describe the different approaches we propose to utilize SCADA log messages and incorporate them into normal behavior models.
In summary, we introduce three different ways for utilizing SCADA log messages: Extracting input features for normal behavior models, Data filtering, and Visualization of warnings.
We will explain each approach in depth.

\subsection{Extracting log embeddings}
  Most machine-learning architectures can only work with vector-shaped numerical inputs. Given that there are limited resources in the research field on how to generate 
  numerical vectors from wind turbine SCADA system logs (see chapter \ref{chap:soa}), we came up with two methods that were proven capable of not only generating embeddings for 
  machine-learning normal behavior models but also improving their accuracy (see chapter \ref{chap:experiments}): 1. our Novel method based on domain knowledge and 
  2. Utilizing an open-source framework for analyzing log data called LogPAI. We will discuss each method in detail.

  \subsubsection{Novel method}
    \label{subsub:novel_method}
    \textbf{Background:}\\
    We scanned through the different log messages available in the dataset looking for information that reflects the turbine state and might help the normal behavior model 
    fit the data more accurately. Since normal behavior models monitor the state of a component by monitoring its temperature, we narrowed the search down to operation and system logs
    that reflect events causing a change of temperature in major components. We, then, ended up with a category of logs that shows the states of internal or external ventilators
    of some components (see table \ref{tab:logs}). Being parts of the cooling systems of major components, fans or ventilators must affect the component's temperature.
    \begin{table}[H]
      \centering
      \begin{tabular}{|c|c|}
      \hline
       \textbf{Log text template} & \textbf{Log text sample}\\
       \hline
       Gen. ext. vent. \_, temp:\_\_\_\degree C & Gen. ext. vent. 2, temp:65\degree C \\
       Gen. int. vent. \_, temp:\_\_\_\degree C & Gen. int. vent. 1, temp:50\degree C \\
       HV Trafo. vent. \_, temp:\_\_\_\degree C & HV Trafo. vent. 0, temp:2\degree C \\
       Nac.vent.\_, nac/gear:\_\_\_/\_\_\_\degree C & Nac.vent.3, nac/gear:43/ 54\degree C \\
      \hline
    \end{tabular}
    \caption{Example log text templates with sample texts}
      \label{tab:logs}
    \end{table}

    Indeed, our analysis showed a clear relationship between the state of a ventilator and the temperature of its turbine component.
    As shown in Fig. \ref{fig:vent}, at low temperatures of the generator bearings, the internal ventilator will switch off. The bearings will then heat up which, in turn, causes
    the ventilator to turn on which cools the bearings down, and so on.

    \begin{figure}[!htbp]
      \begin{center}
        \includegraphics[scale=0.5]{Methods/Vent_Signals.png}
      \end{center}
      \caption{Generator internal vent control signals and their effect on the generator bearings temperature}
      \label{fig:vent}
    \end{figure}

    \begin{flushleft}
      \textbf{Method:}\\
      Analyzing the log texts of interest (e.g., \emph{Gen. ext. vent. 2, temp:65\degree C}), we deduce that they provide three pieces of information: 
      1. Description of the ventilator (e.g., \emph{Gen. ext. vent.}), 2. State of the ventilator (\emph{0, 1, 2 or 3}), 
      3. Temperature of the turbine component the ventilator is installed in (e.g., \emph{65\degree C}).
    \end{flushleft}
    Since the component temperature is regularly provided as a SCADA signal, we decided to focus on the other two parts of the log messages. 
    Our method simply filters log messages containing the word "vent." and creates a new feature for every ventilator (1.) found in the data having its state (2.) as a value.\\
    In contrast to the signals data fixed rate of occurrence (10 min), the generated log embeddings have an inconsistent frequency (the SCADA system creates a new log entry only 
    when a ventilator changes states). We join both datasets by taking the value of the last occurrence in the log embeddings vector within a 10-min window relative to a signal reading. 
    Gaps in the log feature columns in the resulting dataset are then filled by propagating the last valid observation forward to the next valid (a ventilator has the same state as long as it hasn't changed).\\
    Measuring the Kendall correlation factor between the generated log embeddings and all the signals of the turbines, we found that for every temperature signal, there is at least 
    one log embeddings feature that, on average, highly correlates ($Rank>0.5$) with it.

    \begin{figure}[!htbp]
      \begin{center}
        \includegraphics[scale=0.376]{Methods/Log_Merge.png}
      \end{center}
      \caption{Demonstration of the join operation between the signals 10-min dataset and a log embeddings vector}
      \label{fig:log-merge}
    \end{figure}
    

  \subsubsection{Utilizing LogPAI}
    LogPAI (Log Analytics Powered by AI) is a study project and open-source platform for analyzing and managing log data \cite{LogPAI}. 
    Tsinghua University researchers started the project, which focuses on developing efficient algorithms and tools for log analysis, anomaly detection, and log data visualization.
    LogPAI includes a complete suite of log analysis and processing tools such as Logparser, Loglizer, and Logreduce. 
    These applications can assist users in preprocessing and parsing raw log data, detecting anomalies and patterns, and summarizing log data concisely and understandably.
    We decided to utilize LogPAI's Logparser (\cite{Logparser_1}, \cite{Logparser_2}) and Loglizer \cite{Loglizer} to respectively parse and create numerical features from 
    SCADA logs in a more generic and automated way.\\

    \begin{flushleft}
      \textbf{Preprocessing using Logparser:}\\
      From the list of parsers available in the toolkit, we decided to use Drain \cite{Drain} given that it is an online parser, which means it can process the SCADA logs 
      in real-time as they are generated. The Drain algorithm groups similar log messages together and extracts structured events from them using a 
      clustering-based approach. The research demonstrates that Drain is very good at dealing with enormous amounts of log data and extracting meaningful events from 
      noisy and diverse log data. Several phases are involved in the Drain algorithm, including log parsing, log message clustering, and event extraction. 
      Drain employs a fixed-depth tree to parse log messages into a set of log keys and their related values during the log parsing stage. The log keys 
      are unique identifiers for each type of log message, whereas the log values are the specific information connected with each log message. Drain uses a similarity measure 
      to compare the log keys and values of each log message and allocates them to the best appropriate cluster based on their similarity scores during the log message 
      clustering stage. Drain creates a template for each cluster that summarizes the relevant information contained in the log messages once the log messages have been clustered.
      Overall, the Drain algorithm makes an important contribution to log data analysis and management by providing a scalable and effective approach for extracting structured 
      events from unstructured log data.
      Applying Drain on the SCADA log data at hand by specifying its log format "\emph{<TimeDetected>,<TimeReset>,<UnitTitle>,<Content>,<UnitTitleDestination>}", we get 
      output structured log data (see Fig. \ref{fig:logparser} for an example) that the Loglizer can process to generate numerical features.
    \end{flushleft}

    \begin{figure}[!htbp]
      \begin{center}
        \includegraphics[scale=0.349]{Methods/Logparser.png}
      \end{center}
      \caption{Sample raw logs and their corresponding structured logs after being parsed by Drain}
      \label{fig:logparser}
    \end{figure}

    \begin{flushleft}
      \textbf{Creating numerical features using Loglizer}:\\
      Loglizer's \emph{Feature Extraction} component supports various feature extraction techniques, such as Bag-of-Words, TF-IDF, and Word2Vec, 
      to capture the essential information contained in log data. We utilized the Loglizer feature extractor, using TF-IDF \cite{TF-IDF} for term weighting, to generate numerical 
      features from the parsed logs' \emph{Event IDs}. 
    \end{flushleft}

\subsection{Data labeling and filtering}
  In this approach, we developed a method to improve SCADA-data-driven wind turbine power curve models 
  (for a comprehensive review of the various modeling techniques used to predict the power output of wind turbines and their applications in wind-based energy systems, 
  see \cite{Power_curves}). We start by extracting the log messages that report the current state of operation; namely logs containing one of the following expressions:
  \begin{bulletList}
    \item \emph{"Run"},
    \item \emph{"(Stop|Pause).*kW.*RPM"}, or
    \item \emph{"new SERVICE state"}
  \end{bulletList}
  The SCADA signals are then merged with the extracted log messages, using the same join strategy described in \ref{subsub:novel_method}, and labeled based on the following logic:
  \begin{bulletList}
    \item Turbine's state of operation = \emph{"Run"}, if the log feature contains the expression \emph{"Run"} or \emph{"new SERVICE state: 1"}
    \item Turbine's state of operation = \emph{"Stop"}, if the log feature contains the expression \emph{"(Stop|Pause).*kW.*RPM"} or \emph{"new SERVICE state: 0"}
  \end{bulletList}

  \begin{figure}[!htbp]
    \begin{center}
      \includegraphics[scale=0.45]{Methods/power_curve.png}
    \end{center}
    \caption{T01 power curve with log-feature-based labels}
    \label{fig:power_curve}
  \end{figure}

  The log-based feature we introduced showed a clear improvement in the accuracy of power curve models (see experiment..TODO) when used to filter the data being input 
  to the normal behavior model (using data points having \emph{"Run"} as the state of operation exclusively).

\subsection{Visualization of warnings}
  Here, we introduced a straightforward yet effective way of visualizing (e.g., on an operation dashboard) messages from the Alarm and Warning logs that are relevant 
  to faults detected or predicted by normal behavior models and that are worth being reported to the operators.\\
  When the normal behavior model detects a fault in a certain turbine component, the SCADA logs are queried for messages reporting high temperatures in this component 
  during the same time window (e.g., last hour, last 12 hours, current day,\dots). If found, these messages could be included in the system reports that get sent to the operators
  to inform them of the detected failure. This gives more visibility and credibility to the detected/predicted failure by the system.\\
  (TODO add graph showing an example)

\clearpage

\section{Summary}
TODO:
PUSH TO THE TOP\\
Diagram of all methods put together: ML model + log feature + Anomaly detection + Alarms,...
\chapter{Experiments}
\label{chap:experiments}
In this chapter, we describe a set of experiments we ran to quantitatively and qualitatively measure the effect of incorporating the log embeddings introduced into normal behavior models
when applied to both \emph{healthy} and \emph{faulty} turbines. For a normal behavior model monitoring a component of a turbine, we considered this turbine \emph{faulty} if 
a failure was reported, in the failures dataset, related to this specific component of this turbine. It is considered \emph{healthy} if no failures, relating to this turbine's component,
were reported.\\
To compare different models in an identical setup, we use the following metrics:
\begin{bulletList}
    \item \textbf{Root Mean Squared Error ($RMSE$)}: It is a commonly used metric to evaluate the performance of a predictive model or an estimator.
    The $RMSE$ is calculated as the square root of the mean of the squared differences between the predicted ($y_{predicted}$) and actual values ($y_{actual}$), or as follows:
    \begin{equation}
        RMSE = \sqrt{\frac{1}{n} * \sum_{i}^{n} (y_{predicted}^i - y_{actual}^i)^2}
    \end{equation}
    where $n$ is the number of data points in a dataset. The RMSE is expressed in the same units as the original data. 
    As a rule of thumb: The lower the RMSE, the better the model fits the data.
    \item \textbf{Numbers of anomalies and alarms detected during a given period}: We use these numbers to measure the capability of a model to detect/predict a failure. 
    The number of anomalies detected reflects the total number of anomalous data points, whereas the number of alarms detected counts only the number of operation days of a turbine 
    where the system notified the operator of a potential failure by sending an alarm (if the number of anomalies detected in a day exceeds a certain threshold, 
    as explained in \ref{subsub:anvsal}). When compared to another model, we consider a model more \emph{capable} of predicting failures if it 
    detects more anomalies and/or sends more alarms during the time of abnormal operation of a faulty turbine given that it reported no anomalies or alarms during the normal operation 
    of the same turbine. In other words, we compare these metrics between models when applied to the test data of a faulty turbine, assuming that the data used to train these models 
    was collected from the turbine in a period when it was operating in a healthy state; hence no anomalies should be detected in this period.
    \item \textbf{Timestamps of the first anomaly detected and alarm sent}: Used to compare the capability of different models to early-detect failures, when applied to a faulty turbine. The earlier
    the first anomaly is detected or the first alarm is sent the better.
\end{bulletList}
All the condition monitoring normal behavior models used in our experiments were trained to monitor the generator bearings of a turbine (i.e., having the average temperature in the generator bearings 
as a target), whereas the power curve normal behavior models monitor the average power production of a turbine according to the grid (in kW). The input features used are listed in \ref{sub:featselect}.

%Experiment I
\section{Benchmark NBM architecture}
\label{exp:I}

In the early stages of this work, we trained linear regression models due to their lightweight and low computational power needed. Knowing that they are incapable of capturing non-linear
relationships in the data, we assumed that the linear regression models would be outperformed by feed-forward neural networks when it comes down to fitting the 
signals data of a healthy turbine. To test this hypothesis and select a specific architecture to be used as a benchmark NBM model in other experiments, we did this simple experiment 
to compare the $RMSE$ scores of both models.\\
This experiment was conducted on a healthy turbine (Turbine 01). Both models were trained on signals data collected between 01/09/2016 and 30/08/2017 and tested on data collected between
01/09/2017 and 31/12/2017.\\
As shown in Table \ref{tab:Experiment I results}, the feed-forward network outperformed the linear regression model---as expected---and was used as a baseline in all the other experiments.
\begin{table}[H]
        \centering
    \begin{tabular}{|c|c|c|}
    \hline
         \textbf{Metric} & \textbf{Linear regression} & \textbf{Feed-forward network}\\
         \hline
         Training RMSE & 5.29 & 4.86\\
         \hline
         Testing RMSE & 5.80 & 5.78 \\
    \hline
    \end{tabular}
    \caption{Experiment I results: RMSEs measured and used to compare between the benchmark models}
        \label{tab:Experiment I results}
\end{table}

%Experiment II

\section{Effect of incorporating log embeddings into NBM for condition monitoring when applied to a healthy turbine (T01)}
\subsection{Setup}
This experiment aims to quantitatively and qualitatively measure 
the effect of incorporating SCADA-log-based embeddings into the baseline normal behavior model when applied to a healthy turbine.
(see method \ref{sub:dk_method} and \ref{subsub:Loglizer})\\
We ran this experiment three times: one time using the baseline normal behavior model (i.e., using only SCADA signals as input features), 
repeated once after adding the log embeddings generated based on domain knowledge as input features (denoted as \emph{Model-DK}) and 
another time after incorporating LogPAI-generated log embeddings (denoted as \emph{Model-PAI}).
The models were trained on data collected between 01/09/2016 and 30/08/2017 and tested on data collected between
01/09/2017 and 31/12/2017.\\
As shown in Table \ref{tab:expII:domain-knowledge-feats}, all log embeddings generated based on domain knowledge highly correlate with the target feature 
and were hence used as input features in \emph{Model-DK} in addition to the selected SCADA signals.
As for the log embeddings generated using LogPAI, only a few features were found to relatively highly correlate (correlation factor greater than 0.3 or less than -0.3) 
with the target feature (see Table \ref{tab:expII:logpai-feats}) and were selected, in addition to the selected SCADA signals, as input features in \emph{Model-PAI}.
\begin{table}[H]
    \parbox{.45\linewidth}{
    \centering
    \begin{tabular}{|c|c|}
        \hline
        Feature & Correlation\\
        \hline
        \multicolumn{1}{|m{0.25\textwidth}|}{Generator external ventilator} & 0.713057\\
        \hline
        \multicolumn{1}{|m{0.25\textwidth}|}{Generator internal ventilator} & 0.730726\\
        \hline
        \multicolumn{1}{|m{0.25\textwidth}|}{High-voltage transformer ventilator} & 0.513700\\
        \hline
        \multicolumn{1}{|m{0.25\textwidth}|}{Nacelle ventilator} & 0.514112\\
        \hline
    \end{tabular}
    \caption{Measures of Kendall's correlation between the domain knowledge-based log embeddings and the target feature in Turbine 01}
    \label{tab:expII:domain-knowledge-feats}
    }
    \hfill
    \parbox{.45\linewidth}{
    \centering
    \begin{tabular}{|c|c|}
        \hline
        Feature & Correlation\\
        \hline
        \multicolumn{1}{|m{0.25\textwidth}|}{\hl{LogPAI Feature 1}} & \hl{-0.335313}\\
        \hline
        \multicolumn{1}{|m{0.25\textwidth}|}{\hl{LogPAI Feature 2}} & \hl{-0.320031}\\
        \hline
        \multicolumn{1}{|m{0.25\textwidth}|}{LogPAI Feature 3} & 0.015902\\
        \hline
        \multicolumn{1}{|m{0.25\textwidth}|}{LogPAI Feature 4} & -0.220749\\
        \hline
        \multicolumn{1}{|m{0.25\textwidth}|}{LogPAI Feature 5} & 0.083943\\
        \hline
        \multicolumn{1}{|m{0.25\textwidth}|}{\hl{LogPAI Feature 6}} & \hl{-0.303429}\\
        \hline
        \multicolumn{1}{|m{0.25\textwidth}|}{LogPAI Feature 7} & 0.191848\\
        \hline
        \multicolumn{1}{|m{0.25\textwidth}|}{LogPAI Feature 8} & -0.045460\\
        \hline
        \multicolumn{1}{|m{0.25\textwidth}|}{LogPAI Feature 9} & -0.077507\\
        \hline
        \multicolumn{1}{|m{0.25\textwidth}|}{LogPAI Feature 10} & -0.157537\\
        \hline
        \multicolumn{1}{|m{0.25\textwidth}|}{LogPAI Feature 11} & -0.102636\\
        \hline
        \multicolumn{1}{|m{0.25\textwidth}|}{LogPAI Feature 12} & 0.018344\\
        \hline
        \multicolumn{1}{|m{0.25\textwidth}|}{LogPAI Feature 13} & 0.244219\\
        \hline
        \multicolumn{1}{|m{0.25\textwidth}|}{LogPAI Feature 14} & -0.129601\\
        \hline
        \multicolumn{1}{|m{0.25\textwidth}|}{\hl{LogPAI Feature 15}} & \hl{0.361669}\\
        \hline
        \multicolumn{1}{|m{0.25\textwidth}|}{LogPAI Feature 16} & -0.010470\\
        \hline
    \end{tabular}
    \caption{Measures of Kendall's correlation between the log embeddings generated based on the event IDs using the LogPAI framework and the target feature in Turbine 01. Selected features are highlighted.}
    \label{tab:expII:logpai-feats}
    }
    \end{table}

At this stage, the main goal is to test whether those highly-correlating features would improve the baseline model, or, 
they provide redundant information that could be indirectly deduced from the signal features.

\subsection{Results}
\subsubsection{Performance}
As reported in Table \ref{tab:Experiment II results}, the incorporated log embeddings, both in \emph{Model-DK} and \emph{Model-PAI}, improved the RMSE scores of the models. 
This shows that those features provided additional information to the model that wasn't available in the selected SCADA signals, which shows that our proposed methods 
were indeed capable of retrieving valuable information from the SCADA logs.

\begin{table}[H]
    \centering
\begin{tabular}{|c|c|c|c|}
\hline
     \textbf{Metric} & \textbf{\emph{Baseline}} & \textbf{\emph{Model-DK}} & \textbf{\emph{Model-PAI}}\\
     \hline
     Training RMSE & 4.864 & 4.087 & 4.617\\
     \hline
     Testing RMSE & 5.784 & 5.195 & 5.607\\
\hline
\end{tabular}
\caption{Experiment results: RMSEs measured and used to compare between the \emph{Baseline} model, \emph{Model-DK} and \emph{Model-PAI} when applied to Turbine 01}
    \label{tab:Experiment II results}
\end{table}

\subsubsection{Anomaly detection}
Applying the models to the testing dataset, 12, 15 and 8 data points were labeled anomalous by the \emph{Baseline} model,
\emph{Model-DK} and \emph{Model-PAI}, respectively. Whether to consider these data points as false positives or not is 
arguable. One could consider them as false positives based on the premise that the turbine was operating in a healthy state.
In this case, \emph{Model-PAI} would be ranked highest in terms of anomaly detection. However, these data points could indeed
be anomalous and be signaling a fault that might happen in the future. Here, \emph{Model-DK} would show a higher sensitivity 
to anomalous data points. \\
Given that the dataset provided doesn't include data from the following years (2018+), 
we weren't able to test the different possibilities and will leave the interpretation open to the reader.\\

No alarms were reported in all three models. This result shows that none of the anomalies detected was considered 
critical enough for the operator to be notified, which aligns with the assumption that the turbine is healthy.




%Experiment III

\section{Effect of incorporating log embeddings into NBM for condition monitoring when applied to a faulty turbine (T09)}
\subsection{Setup}
This experiment aims to quantitatively and qualitatively measure 
the effect of incorporating SCADA-log-based embeddings into the baseline normal behavior model when applied to a faulty turbine 
(see method \ref{sub:dk_method}, \ref{subsub:vis_warnings} and \ref{subsub:Loglizer}).\\
For this turbine (T09), several failures relating to the generator bearings were reported starting on 07/06/2016 and ending on 17/10/2016
by replacing the generator bearings (see Table \ref{tab:failures}). In this case, not only do we want to measure the performance of our models fitting the data 
during the presumably healthy (training) period, but also we want to test their capability of detecting the recorded failures early and notifying the operator 
in cases of major anomalies in the operation of the turbine's component.\\
We ran this experiment three times: one time using the baseline normal behavior model (i.e., using only SCADA signals as input features), 
repeated once after adding the log embeddings generated based on domain knowledge as input features (denoted as \emph{Model-DK}) and 
another time after incorporating LogPAI-generated log embeddings (denoted as \emph{Model-PAI}).
The models were trained on data collected between 01/01/2016 and 15/02/2016 and tested on data collected between
16/02/2016 and 18/10/2016. Due to the limits of the dataset at hand, only two and a half months of data could be used to train the models since, 
according to our analysis, this is the period where the turbine was assumably still operating in healthy conditions.
Due to the shortage in the training dataset, some features in the testing dataset were found to be out-of-distribution (see Fig. \ref{fig:Boxplot_T09}). 
This fact made the interpretation of the results a bit more challenging, it, however, helped test the robustness of the 
different models' architectures.

\begin{figure}[H]
    \begin{center}
      \includegraphics[scale=0.43]{Experiments/Boxplot_T09.png}
    \end{center}
    \caption{Boxplots showing the distribution of signals collected from Turbine 09 sensors used to train and test the models}
    \label{fig:Boxplot_T09}
  \end{figure}

As shown in Table \ref{tab:expIII:domain-knowledge-feats}, all log embeddings generated based on domain knowledge highly correlate with the target feature 
and were hence used as input features in \emph{Model-DK} in addition to the selected SCADA signals.
As for the log embeddings generated using LogPAI, only a few features were found to relatively highly correlate (correlation factor greater than 0.3 or less than -0.3) 
with the target feature (see Table \ref{tab:expIII:logpai-feats}) and were selected, in addition to the selected SCADA signals, as input features in \emph{Model-PAI}.
\begin{table}[H]
    \parbox{.45\linewidth}{
    \centering
    \begin{tabular}{|c|c|}
        \hline
        Feature & Correlation\\
        \hline
        \multicolumn{1}{|m{0.25\textwidth}|}{Generator external ventilator} & 0.505995\\
        \hline
        \multicolumn{1}{|m{0.25\textwidth}|}{Generator internal ventilator} & 0.656839\\
        \hline
        \multicolumn{1}{|m{0.25\textwidth}|}{High-voltage transformer ventilator} & 0.500316\\
        \hline
        \multicolumn{1}{|m{0.25\textwidth}|}{Nacelle ventilator} & 0.480353\\
        \hline
    \end{tabular}
    \caption{Measures of Kendall's correlation between the domain knowledge-based log embeddings and the target feature in Turbine 09}
    \label{tab:expIII:domain-knowledge-feats}
    }
    \hfill
    \parbox{.45\linewidth}{
    \centering
    \begin{tabular}{|c|c|}
        \hline
        Feature & Correlation\\
        \hline
        \multicolumn{1}{|m{0.25\textwidth}|}{LogPAI Feature 1} & -0.279547\\
        \hline
        \multicolumn{1}{|m{0.25\textwidth}|}{LogPAI Feature 2} & 0.026935\\
        \hline
        \multicolumn{1}{|m{0.25\textwidth}|}{\hl{LogPAI Feature 3}} & \hl{-0.320831}\\
        \hline
        \multicolumn{1}{|m{0.25\textwidth}|}{LogPAI Feature 4} & 0.019996\\
        \hline
        \multicolumn{1}{|m{0.25\textwidth}|}{LogPAI Feature 5} & -0.296431\\
        \hline
        \multicolumn{1}{|m{0.25\textwidth}|}{LogPAI Feature 6} & 0.018961\\
        \hline
        \multicolumn{1}{|m{0.25\textwidth}|}{LogPAI Feature 7} & 0.231462\\
        \hline
        \multicolumn{1}{|m{0.25\textwidth}|}{LogPAI Feature 8} & -0.214312\\
        \hline
        \multicolumn{1}{|m{0.25\textwidth}|}{LogPAI Feature 9} & 0.172929\\
        \hline
        \multicolumn{1}{|m{0.25\textwidth}|}{\hl{LogPAI Feature 10}} & \hl{0.324340}\\
        \hline
        \multicolumn{1}{|m{0.25\textwidth}|}{LogPAI Feature 11} & -0.131364\\
        \hline
        \multicolumn{1}{|m{0.25\textwidth}|}{LogPAI Feature 12} & -0.011957\\
        \hline
        \multicolumn{1}{|m{0.25\textwidth}|}{LogPAI Feature 13} & -0.080340\\
        \hline
        \multicolumn{1}{|m{0.25\textwidth}|}{LogPAI Feature 14} & -0.158273\\
        \hline
        \multicolumn{1}{|m{0.25\textwidth}|}{LogPAI Feature 15}& -0.126988\\
        \hline
        \multicolumn{1}{|m{0.25\textwidth}|}{LogPAI Feature 16} & -0.074183\\
        \hline
    \end{tabular}
    \caption{Measures of Kendall's correlation between the log embeddings generated based on the event IDs using the LogPAI framework and the target feature in Turbine 09. Selected features are highlighted.}
    \label{tab:expIII:logpai-feats}
    }
    \end{table}

At this stage, the main goal is to test whether the generated log embeddings would also improve the performance of the baseline model when applied to a 
faulty turbine. In addition to that, we want to test the effect of these features on improving the model's capability of failure early detection.
For that, we start by measuring the RMSE scores to compare the models' performances and then analyze the anomalies detected in the testing period 
and the yielded alarms.

\subsection{Results}
\subsubsection{Performance}
Again, it was shown that the incorporated log embeddings, both in \emph{Model-DK} and \emph{Model-PAI}, improved the RMSE scores of the models (see Table \ref{tab:Experiment III results}). 

\begin{table}[H]
    \centering
\begin{tabular}{|c|c|c|c|}
\hline
     \textbf{Metric} & \textbf{\emph{Baseline}} & \textbf{\emph{Model-DK}} & \textbf{\emph{Model-PAI}}\\
     \hline
     Training RMSE & 8.562 & 8.081 & 8.386\\
     \hline
     Testing RMSE & 9.084 & 8.936 & 9.029\\
\hline
\end{tabular}
\caption{Experiment results: RMSEs measured and used to compare between the \emph{Baseline} model, \emph{Model-DK} and \emph{Model-PAI} when applied to Turbine 09}
    \label{tab:Experiment III results}
\end{table}

\subsubsection{Anomaly \& Early fault detection}
In total, 48, 217 and 236 anomalies were reported by the \emph{Baseline} model, \emph{Model-DK} and \emph{Model-PAI}, respectively.
Those anomalous data points were spread over 29 days for the \emph{Baseline} model, 42 days for \emph{Model-DK} and 53 days for \emph{Model-PAI}.
From these anomalous days, relevant alarm and warning logs were found (see \ref{subsub:vis_warnings} on how these logs are retrieved) on 18 days for both \emph{Model-DK} and \emph{Model-PAI} and only on 14 days for \emph{Baseline} model.
In addition to that, both \emph{Model-DK} and \emph{Model-PAI} detected the first anomaly one day earlier than the \emph{Baseline} model (16/02/2016 versus 17/02/2016).

This result only shows the major effect of the log embeddings---whether the ones generated by applying domain knowledge or using LogPAI---on the behavior of 
the model during unhealthy conditions of the turbine. The way we interpret this result is as follows: By adding supplementary information regarding the turbine's control signals
found in the SCADA log, the model could provide a better simulation of the turbine in healthy conditions and, hence, detect abnormalities more easily
during unhealthy states of operation.\\
Using the method described in \ref{subsub:anvsal}, only \emph{Model-DK} sent alarms with a total of six alarms starting on 19/02/2016 and ending on 23/08/2016.
This is due to the fact that the peak number of anomalies detected per day was significantly high for \emph{Model-DK} (between 13 and 28) 
compared to \emph{Model-PAI} (12) and the \emph{Baseline} model (3 only). 
On one hand, we believe this result is due to the limited sample size used to train the model. 
However, on the other hand, it shows the higher robustness of \emph{Model-DK}: the limited training dataset was enough for the model 
to report higher numbers of anomalies detected per day during the validation period, high enough for it to trigger an alarm.\\
A summary of the discussed results is shown in Table \ref{tab:summary_expIII}.

\begin{table}[H]
    \centering
    \begin{tabular}{|c|c|c|c|}
        \hline
            \textbf{Metric} & \textbf{\emph{Baseline}} & \textbf{\emph{Model-DK}} & \textbf{\emph{Model-PAI}}\\
            \hline
            \#Anomalous data points & 48 & 217 & 236\\
            \hline
            ..firstly detected on & 17/02/2016 & 16/02/2016 & 16/02/2016\\
            \hline
            ..lastly detected on & 29/09/2016 & 06/10/2016 & 11/10/2016\\
            \hline
            \#Anomalous days & 29 & 42 & 53\\
            \hline
            ..of which warning logs were found & 14 & 18 & 18\\
            \hline
            \#Alarms sent & 0 & 6 & 0\\
            \hline
            ..firstly on & \- & 19/02/2016 & \-\\
            \hline
            ..lastly on & \- & 23/08/2016 & \-\\
        \hline
    \end{tabular}
    \caption{Anomaly and early fault detection: Summary of experiment results}
    \label{tab:summary_expIII}
\end{table}

\section{Effect of log-based data filtering on NBM for power curve modeling}
    \label{sec:Experiment IV}
    \subsection{Setup}
    The main goal of this experiment is to test the effectiveness of the method \ref{subsub:PC} in improving normal behavior models 
    having power production as the target, also known as power curve models. To do so, we trained a normal behavior model on 
    data collected from Turbine 01 sensors between September 2016 and August 2017 and tested on data collected between September 2017 
    and December 2017. We used only the ambient wind speed (m/s) and ambient temperature (\degree C) signals as input features 
    and the average power production according to the grid (kW) as the target feature.\\
    We trained three different models having the same architecture but using different datasets: 
    Using all the raw signals collected; denoted as \emph{Baseline}, 
    using raw signals when the turbine was spinning only (i.e., rotor speed greater than zero); denoted as \emph{Model-Spin},
    and using raw signals when the turbine was operating in a "Run" state only based on the SCADA log; denoted as \emph{Model-Run}.\\

    Figure \ref{fig:labels} shows how the data points are labeled based on the filters explained. One could see that \emph{Model-Spin} 
    doesn't consider all data points where the turbine isn't spinning (red and yellow points) including when it's in a "Run" state
    but the wind speed hasn't reached the cut-in speed yet (yellow points). Although \emph{Model-Run} will also not consider some of the data points
    where the turbine is not spinning (red points), it will still consider the yellow points. In addition to that, it will exclude data points 
    where the turbine is in a "Stop" state but still spinning (most probably still in the transition phase after gradually applying the brakes).
    The \emph{Baseline} is simply trained on all the data points.

    \begin{figure}[H]
        \begin{center}
          \includegraphics[scale=0.45]{Experiments/labels.png}
        \end{center}
        \caption{Turbine 01 power curve: Data points labeled based on different filters}
        \label{fig:labels}
      \end{figure}
    
      \subsection{Results}
      Here, we simply compared the models' performances by measuring their RMSE scores. Comparing the models' anomaly detection 
      and performance evaluation capabilities wasn't in the scope of this work, as we mainly focused on analyzing the generator-related condition 
      monitoring models. We did however reach these results while analyzing different strategies to utilize the SCADA logs in the 
      context of SCADA-based condition monitoring and decided they were worth documenting.\\

      Table \ref{tab:Experiment IV results} shows the measured RMSE scores of the three models. As shown, filtering the data improved
      the model's performance drastically. The filtering based on the turbine's state retrieved from the SCADA logs (\emph{Model-Run}) 
      yielded the best results in this setting. This shows that the generated labels provide valuable insights into the turbine's state 
      and can be used instead of a simple filter by the rotor's speed, especially if the latter isn't available as a SCADA signal for a given turbine.
      Figure \ref{fig:3pcs} also shows the better power curve fit both \emph{Model-Spin} and \emph{Model-Run} have compared to the \emph{Baseline} model 
      when applied to the testing dataset.

      \begin{table}[H]
        \centering
    \begin{tabular}{|c|c|c|c|}
    \hline
         \textbf{Metric} & \textbf{\emph{Baseline}} & \textbf{\emph{Model-Spin}} & \textbf{\emph{Model-Run}}\\
         \hline
         Training RMSE & 179.200 & 61.707 & 51.309\\
         \hline
         Testing RMSE & 139.972 & 59.988 & 50.251\\
    \hline
    \end{tabular}
    \caption{Experiment results: RMSEs measured and used to compare between the \emph{Baseline} model, \emph{Model-Spin} and \emph{Model-Run} when applied to Turbine 01}
        \label{tab:Experiment IV results}
    \end{table}

    \begin{figure}[H]
        \begin{center}
          \includegraphics[scale=0.5]{Experiments/3pcs.png}
        \end{center}
        \caption{True versus predicted power curves of the \emph{Baseline} model, \emph{Model-Spin} and \emph{Model-Run}, respectively, 
        when applied to the testing dataset of Turbine 01}
        \label{fig:3pcs}
      \end{figure}
    


\chapter{Conclusions and Future Works}
\label{chap:conclusions}

\section{Conclusions}

In this study, titled "Utilizing SCADA-Log Data to Improve Normal Behavior Models for Wind Turbine Condition Monitoring," 
we aimed to leverage SCADA event log data to enhance the performance of condition-monitoring machine learning models, 
improve anomaly detection, and provide valuable insights for wind turbine operation and maintenance. 
This section presents the key findings and contributions of the research. \par

Firstly, we successfully generated log embeddings based on the SCADA event log, which served as informative input features 
for the condition monitoring machine learning models. The utilization of these log embeddings led to significant 
improvements in the models' fitting to the target feature, enhancing their ability to detect anomalies. 
The incorporation of SCADA-log data as input features demonstrated its effectiveness in improving the performance of 
wind turbine condition monitoring systems. Two methods were employed to generate log embeddings as input features for machine learning models. \\
The first method utilized domain knowledge, incorporating expert insights and understanding of wind turbine operations to develop the log embeddings. 
This approach capitalized on the specific characteristics and patterns present in the SCADA event log data, allowing for an accurate representation of normal behavior. \\
The second method involved the utilization of an open-source framework known as LogPAI, which facilitated the generation of log embeddings. 
By employing this framework, the study leveraged existing resources and techniques, benefiting from the advancements in log analysis and embedding generation. 
\par

Furthermore, we implemented a method for labeling and filtering data used in power curve models, leveraging the SCADA event log. 
By incorporating the insights gained from the log data, we were able to optimize the accuracy and reliability of power curve models, 
contributing to more accurate predictions of wind turbine performance and improved decision-making processes in wind farm operations. \par

In addition to the improvements in machine learning models and power curve modeling, we also developed a simple operation dashboard that visualizes 
relevant warnings and alarms extracted from the SCADA event log. 
This dashboard provides a comprehensive overview of the turbine's operational status and facilitates the timely identification of potential issues. 
By highlighting relevant events and trends, it empowers operators to make informed decisions and take proactive measures to ensure the optimal performance 
and maintenance of wind turbines. \par

Overall, the findings of this study demonstrate the significant potential of utilizing SCADA event log data for wind turbine condition monitoring. 
The integration of log embeddings as input features enhances the performance of machine learning models, while the utilization of SCADA data in power curve modeling 
improves the accuracy of performance predictions. 
Additionally, the developed operation dashboard allows for effective visualization and monitoring of relevant alarms and warnings. 
These contributions collectively contribute to the advancement of wind turbine monitoring and maintenance practices, ultimately leading to increased operational efficiency and reduced downtime. \par

The insights gained from this research lay the foundation for future studies in the field of wind turbine condition monitoring. 
Further exploration of the potential of SCADA-log data integration, optimization of machine learning models, and the development of advanced visualization techniques can foster continued advancements in the industry.

\section{Future Works}
\label{sec:future_works}

The preceding research has provided valuable insights into the utilization of SCADA-log data for improving normal behavior models in wind turbine condition monitoring. 
However, several avenues for future exploration and development could contribute to further advancements in this field. 
This section presents potential areas of focus for future research, aiming to expand the current knowledge and address the existing gaps. 
By considering these future works, researchers and practitioners can continue to enhance the accuracy, robustness, and applicability of condition-monitoring techniques for wind turbines.

\subsection{Expanding Dataset Size for Enhanced Model Training and Generalization}

An important aspect to consider in future research is the expansion of the dataset used for training the condition monitoring models. 
While the dataset utilized in this work provided valuable insights and yielded promising results, its size was inherently limited. 
Therefore, one potential avenue for future exploration involves gathering a larger and more diverse dataset encompassing a wider range of wind turbine operational scenarios. \\

By incorporating a larger dataset, the machine learning models can potentially capture a more comprehensive representation of the normal behavior patterns and variations present in wind turbine operations. 
This increased data volume would allow for more robust training, improved generalization, and enhanced anomaly detection capabilities. \\

To obtain a larger dataset, collaborative efforts among industry stakeholders, research institutions, and wind farm operators may be necessary. 
Sharing anonymized SCADA-log data across multiple sites and leveraging international collaborations could facilitate the creation of a more extensive and representative dataset. \\

Furthermore, the availability of a larger dataset would enable the evaluation and comparison of the proposed methods and techniques on a broader scale. 
Robustness and reliability assessments could be conducted across multiple wind farm installations, considering variations in turbine models, geographical locations, and environmental conditions. 
This would provide a more comprehensive understanding of the efficacy and potential limitations of the developed models and algorithms. \\

Therefore, future research endeavors should prioritize the acquisition and utilization of larger datasets to refine and validate the proposed condition monitoring models. 
By encompassing a broader range of operational scenarios, the models can be further optimized and their performance can be thoroughly evaluated, ultimately enhancing their practical applicability in real-world wind turbine condition monitoring systems.

\subsection{Unveiling Deeper Insights through Domain-Knowledge-Based Analysis of SCADA-Log Data}
Moreover, the domain-knowledge-based method employed for extracting log embeddings presents an opportunity for further analysis and extraction of additional insights from the SCADA-log data. 
By delving deeper into the domain knowledge and refining the embedding extraction process, it is possible to uncover more intricate patterns and correlations within the data. 
Future research should consider exploring advanced techniques, such as feature engineering or domain-specific preprocessing, to enhance the information captured by the log embeddings and extract deeper insights that can contribute to a more comprehensive understanding of wind turbine behavior and performance.

\subsection{Incorporating Grid Curtailment Information to Enhance Labeling and Filtering in Power Curve Models}
\label{sub:Curtailment}
In addition, the domain-knowledge-based method employed for labeling and filtering data points in the application of power curve models 
can be further enhanced by incorporating valuable information about grid curtailment, 
which can be extracted from the SCADA log.
By considering grid curtailment events, characterized by instances where the turbine's power output is deliberately limited due to grid constraints or other external factors, 
the labeling and filtering process can be refined to account for these specific conditions. \\

This additional information from the SCADA log about grid curtailment events can contribute to a more precise identification and 
classification of data points in the power curve models. For example, the log event "External power ref.:1392kW" shows that the turbine 
was curtailed and can only produce up to 1392 kW (instead of 2000 kW).
By factoring in the influence of grid curtailment, the models can better capture the true normal behavior of the wind turbine under varying operational conditions, 
resulting in improved accuracy and reliability.

\subsection{Extending Methodology to Monitor Other Wind Turbine Components}
Additionally, it is crucial to explore the applicability of the developed methods to components beyond the generator bearings. 
While this study focused on the condition monitoring of generator bearings, there are other critical components within wind turbines that require monitoring for efficient and reliable operation. 
Future research should aim to test and validate the proposed methods on various turbine components, such as gearbox, pitch system, yaw system, or tower structural elements, to assess their effectiveness and adaptability in detecting anomalies and predicting failures across the entire turbine system.

\subsection{Potential Application of Fuzzy Systems as Normal Behavior Models}
\label{sub:fuzzy}

Furthermore, as part of future work, the exploration of fuzzy systems as alternative normal behavior models holds significant promise. 
Fuzzy systems have demonstrated their effectiveness in capturing and representing complex and uncertain relationships in various domains \cite{SCADA_NBM_Review}. 
By incorporating fuzzy logic and linguistic variables, these systems can provide a flexible and interpretable framework for modeling the normal behavior of wind turbines based on SCADA-log data. \\

The utilization of fuzzy systems in wind turbine condition monitoring could offer advantages such as robustness to data variations and the ability to handle imprecise or incomplete information. 
By integrating fuzzy systems into the existing framework, researchers and practitioners can further improve the accuracy and reliability of anomaly detection algorithms, enhancing the overall performance of condition monitoring systems. \\

Exploring the feasibility and potential benefits of incorporating fuzzy systems as normal behavior models represents an intriguing avenue for future research in this field. 
This would involve designing appropriate fuzzy rule sets, membership functions, and inference mechanisms tailored specifically to the characteristics of SCADA-log data and wind turbine operations. 
Such investigations could deepen our understanding of the strengths and limitations of fuzzy systems in this context and contribute to the advancement of wind turbine condition monitoring techniques. \\

Therefore, investigating the application of fuzzy systems as normal behavior models stands as an important direction for future research, with the potential to further enhance the effectiveness and reliability of condition monitoring approaches in the wind energy industry.

\subsection{Applicability of Methods to Other SCADA-Enabled Systems}

Finally, the methods developed in this study hold the potential for application beyond the specific domain of wind turbine condition monitoring. 
Given the widespread use of SCADA systems across various industries, these methods can be tested and adapted to other systems that employ SCADA for data collection and monitoring purposes.

\appendix

\chapter{Wind Turbine Characteristics}
\label{chap:appendix1}

Here, we list the characteristics of the turbines whose data was used in our experiments.
The following table shows the technical characteristics of these Vestas turbines:

\begin{table}[H]
    \centering
    \begin{tabular}{|c|c|c|c|}
    \hline
    \multirow{5}{*}{\rotatebox[origin=c]{90}{Power}} & Rated power (kW) & 2,000 \\
    & Cut-in wind speed (m/s) & 4 \\
    & Rated wind speed (m/s) & 12  \\
    & Cut-out wind speed (m/s) & 25 \\
    & Wind class (IEC) & IEC II (7.5 - 8.5 m/s) \\
    \hline
    \multirow{7}{*}{\rotatebox[origin=c]{90}{Rotor}} & Diameter (m) & 90 \\
    & Swept area (m\textsuperscript{2}) & 6,362 \\
    & Number of blades & 3  \\
    & Rotor speed, max (rpm) & 14.9 \\
    & Tip speed (m/s) & 70 \\
    & Power density 1 (W/m\textsuperscript{2}) & 314.4 \\
    & Power density 2 (m\textsuperscript{2}/kW) & 3.2 \\
    \hline
    \multirow{4}{*}{\rotatebox[origin=c]{90}{Gearbox}} & & \\
    & Type & Planetary/spur \\
    & Stages (m/s) & 3 \\
    & & \\
    \hline
    \multirow{4}{*}{\rotatebox[origin=c]{90}{Generator}} & Type & Asynchronous \\
    & Speed, max (rpm) & 2,016 \\
    & Voltage (V) & 690  \\
    & Grid frequency (Hz) & 50 \\
    \hline
    \multirow{4}{*}{\rotatebox[origin=c]{90}{Tower}} & Hub height (m) & 80 \\
    & Type & Steel tube \\
    & Shape & Conical  \\
    & Corrosion protection & Painted \\
    \hline
    \end{tabular}
    \caption{Wind turbine characteristics}
    \label{tab:characteristics}
\end{table}

The following graph demonstrates the turbines' power curve provided by the manufacturer:

\begin{figure}[H]
    \begin{center}
      \includegraphics[scale=0.6]{Appendix1/Power_curve.png}
    \end{center}
    \caption{Power curve of the Vestas turbines at an air density of 1.225 kg/m\textsuperscript{3}}
  \end{figure}
\chapter{Recorded Failures}
\label{chap:appendix2}
Here, we show the full list of failures detected in all five turbines and recorded by the technicians or the service company.
These failures were found and documented upon on-site inspections and were used to validate the capability of the normal behavior models to 
predict them early (enough) before actually occurring. Table \ref{tab:failures} shows the full list of failures.

\begin{table}[h]
    \centering
    \begin{tabular}{|c|c|c|c|}
    \hline
    \textbf{Turbine} & \textbf{Component} & \textbf{Recorded at} & \textbf{Technician Remarks} \\
    \hline
    \multirow{2}{*}{01} & Gearbox & 2016-07-18 02:10:00 & Gearbox pump damaged \\
    & Transformer & 2017-08-11 13:14:00 & Transformer fan damaged \\
    \hline
    \multirow{9}{*}{06} & Hydraulic group & 2016-04-04 18:53:00 & Error in pitch regulation \\
    & Generator & 2016-07-11 19:48:00 & Generator replaced \\
    & Generator & 2016-07-24 17:01:00 & Generator temperature sensor failure \\
    & Generator & 2016-09-04 08:08:00 & High temperature generator error \\
    & Generator & 2016-10-02 17:08:00 & Refrigeration system and \\
    & & & temperature sensors in generator replaced \\
    & Generator & 2016-10-27 16:26:00 & Generator replaced \\
    & Hydraulic group & 2017-08-19 09:47:00 & Oil leakage in Hub \\
    & Gearbox & 2017-10-17 08:38:00 & Gearbox bearings damaged \\
    \hline
    \multirow{9}{*}{07} & Generator bearings & 2016-04-30 12:40:00 & High temperature in generator bearing \\
    & & & (replaced sensor) \\
    & Transformer & 2016-07-10 03:46:00 & High temperature transformer \\
    & Transformer & 2016-08-23 02:21:00 & High temperature transformer. \\
    & & & Transformer refrigeration repaired \\
    & Hydraulic group & 2017-06-17 11:35:00 & Oil leakage in Hub \\
    & Generator bearings & 2017-08-20 06:08:00 & Generator bearings damaged \\
    & Generator & 2017-08-21 14:47:00 & Generator damaged \\
    & Hydraulic group & 2017-10-19 10:11:00 & Oil leakage in Hub \\
    \hline
    \multirow{7}{*}{09} & Generator bearings & 2016-06-07 16:59:00 & High temperature generator bearing \\
    & Generator bearings & 2016-08-22 18:25:00 & High temperature generator bearing \\
    & Gearbox & 2016-10-11 08:06:00 & Gearbox repaired \\
    & Generator bearings & 2016-10-17 09:19:00 & Generator bearings replaced \\
    & Generator bearings & 2017-01-25 12:55:00 & Generator bearings replaced \\
    & Hydraulic group & 2017-09-16 15:46:00 & Pitch position error related GH \\
    & Gearbox & 2017-10-18T08:32:00 & Gearbox noise \\
    \hline
    \multirow{4}{*}{11} & Generator & 2016-03-03 19:00:00 & Electric circuit error in generator \\
    & Hydraulic group & 2016-10-17 17:44:00 & Hydraulic group error in the brake circuit \\
    & Hydraulic group & 2017-04-26 18:06:00 & Hydraulic group error in the brake circuit \\
    & Hydraulic group & 2017-09-12 15:30:00 & Hydraulic group error in the brake circuit \\
    \hline
    \end{tabular}
    \caption{List of failures recorded found in the EDP dataset}
    \label{tab:failures}
\end{table}


\bibliographystyle{ThesisStyle}
\bibliography{Thesis}

%\printnomenclature

\end{document}
