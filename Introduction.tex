\chapter{Introduction}
\label{chap:intro}
\minitoc

\section{Background}
In 2020, renewable energy represented 22.1\% of energy consumed in the EU \cite{Renewable_energy_statistics}. This percentage is 
expected to increase drastically in the upcoming years with the target, set by
the European Commission, of at least 32\% by the year 2030 \cite{Renewable_energy_targets}. With the increasing number
of renewable energy assets being deployed every year, automated condition monitoring
solutions are needed for operators to be able to scale up their portfolio of assets.
Monitoring wind turbine health and performance is critical for early fault detection, maintenance planning, and 
optimizing wind farm operations. 
\par Several manufacturers have created so-called condition monitoring systems (CMS). 
These monitor a variety of essential metrics such as drive train vibration, oil quality, and temperatures in some of 
the main components. Such devices are typically deployed as an addition to the regular wind turbine design.
Even though the financial value of CMS's early defect detection has been demonstrated \cite{CMS}, 
their high prices \cite{CMS_Costs} have discouraged operators from implementing them.
Most of the utility-scale wind turbines come, however, with a Supervisory Control and Data Acquisition (SCADA) system by default.
SCADA systems provide significant insights into wind turbine operational behavior. 
They record various types of data related to 
the operation and performance of the turbine which can be divided into two main categories: SCADA \emph{signals} and SCADA \emph{logs}.
The SCADA signals provide real-time readings collected from various sensors installed in the turbine that reflect the current state of operation 
in terms of power production, wind speed, rotor speed, component temperatures,\dots The frequency and the number of signals provided by the SCADA system 
vary based on the turbine's manufacturer, model and technology. The SCADA logs, on the other hand, capture alarms and events recorded by the SCADA 
system in the form of text in a non-fixed frequency.
Some approaches utilize both CMS and SCADA data to perform condition monitoring tasks (e.g., \cite{CMSSCADA}), however, 
several other approaches for condition monitoring were developed in recent years that rely solely on the
SCADA data, given its low cost as it normally doesn't require additional hardware installation. For a comprehensive review 
of different methods for wind turbine condition monitoring using SCADA data, see \cite{SCADA_NBM_Review}.

\par One of the methods used for condition monitoring using SCADA data is Normal Behavior
Modeling (NBM). NBM uses the idea of detecting anomalies from normal operation by
empirically modeling a measured parameter, used to reflect the condition of a specific part of
the turbine, based on a training phase (usually during a healthy state of the turbine). During
operation, the difference between the measured and the modeled/predicted signal is used as an
indicator for a possible fault. A difference of 0, with some tolerance, reflects normal conditions,
whereas a difference greater or less than 0 reflects changed conditions or failures. Utilizing the signals provided 
by the SCADA systems in normal behavior models were proven capable of (early) detecting failures in wind turbines.
For instance, both Zhang et al. \cite{Zhang_Wang} and Bangalore et al. \cite{Bangalore_1} applied machine learning techniques to SCADA signals data to develop anomaly detection models 
for fault diagnosis and prediction of the gearbox bearings of wind turbines. 
The results demonstrate that their condition-monitoring approaches are capable of indicating damage in the components being monitored in advance.

\par Other methods focus on utilizing logs to detect anomalies in the underlying system. Some of these approaches are general-purpose 
and can be applied to any computer system. For example, Brown et al. \cite{RNN} developed recurrent neural network (RNN) language models augmented with attention 
for anomaly detection in system logs that are generally applicable to any computer system and logging source. 
Similarly, Lyu et al. \cite{LogPAI} developed an open-source framework with a set of tools that can be used for automated log parsing, 
anomaly detection and impactful problem identification using machine learning.
Other approaches specifically focus on leveraging the SCADA logs, especially alarms, to detect anomalies in wind turbines.
Rahman et al. \cite{sequence} use rare sequential pattern mining\footnote{Rare sequential pattern mining is a data mining technique used to discover infrequent patterns in sequential data.} 
to find anomalies in SCADA networks. 
They suggest that because anomalous events occur rarely in a system and the architecture and actions of SCADA systems do not change frequently, 
some anomalies can be found via uncommon sequential pattern mining. This anomaly detection might be useful for detecting intrusions or erroneous system behaviors.
Andrade et al. \cite{Cluster} proposed methods that utilize unsupervised machine-learning clustering techniques to profile alarm patterns, identify abnormal events, and improve the detection of anomalies in 
industrial processes in the context of network operators and grid outages.


\section{Motivation \& Objectives}
While both SCADA signals and logs are leveraged in both areas of study, both approaches are performed in isolation, which leaves 
operators with only two options: Either use one condition monitoring system or have to rely on multiple sources of information 
to make informed operational decisions.
\par In one of his recent publications, Letzgus \cite{Letzgus_Log} presented methods from the Natural Language Processing (NLP) domain 
that help to find meaningful representations of SCADA log messages and sequences. 
His methods encode messages from the SCADA log into vector representation by creating one-hot vectors or applying the 
Correlated Occurrence Analogue to Lexical Semantic with temporal information (COALS-t) algorithm. 
This permits and facilitates the successful implementation of machine-learning-based condition monitoring models.
\par Additionally, Leahy et al. \cite{Leahy} demonstrated a method to label the SCADA signals by identifying stoppages that occurred in the turbines 
and that are recorded as alarms in the SCADA log. This data was then used to perform fault detection using classification techniques.

% Given that, we developed methods that refine the normal behavior models, 
% trained on SCADA signals, by leveraging the wealth of information within the SCADA log data.\\
\par Inspired by the work done in the literature, the primary objective of this thesis is to bring both "worlds" of SCADA signals- and log-based anomaly detection together 
by incorporating SCADA log data into wind turbine condition monitoring models. 
By mining SCADA log data, we aim to identify subtle patterns, correlations, and dependencies that may contain information about operation conditions or 
control events which could help improve the accuracy and robustness of normal behavior models in case of events unexplainable by the SCADA signals. 
Furthermore, the utilization of advanced machine learning algorithms, such as deep learning and anomaly detection techniques, will enhance the detection and prediction capabilities for abnormal behaviors and potential faults.
We present different methods to utilize the SCADA logs and incorporate them into machine-learning normal behavior models by generating SCADA log-based vectors that can be used 
as input features and labels that can be used to filter the SCADA signals being fed into the models. In addition to that, we propose a simple-yet-effective method to visualize 
relevant alarms and warnings found in the SCADA logs which encourages operators to deal with one system only to monitor the state of the turbines.

\par The findings of this research are expected to contribute to the field of wind turbine condition monitoring by providing enhanced techniques for normal behavior modeling. 
The developed models can serve as a basis for effective fault detection, condition assessment, and predictive maintenance strategies, ultimately leading to increased reliability, reduced downtime, and improved operational efficiency of wind turbines.

\par Chapter 2 will provide an overview of the methodology employed, along with the relevant literature, on wind turbine condition monitoring, SCADA-log data analysis, and machine learning techniques 
(including data preprocessing and feature extraction) applied to wind turbine condition monitoring. 
Chapter 4 will showcase the experimental results and discuss the performance of the proposed models. 
Finally, Chapter 5 will summarize the findings, draw conclusions, and provide recommendations for future research.



